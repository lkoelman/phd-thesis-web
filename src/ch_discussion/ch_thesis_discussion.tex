
%
%
%
%
%
%
%
%
%

%
%
%

%
%
%
%
%
%
%
%
%
%
%
%
%

%
%
%
%
%
%
%

%
%
%
%
%
%
%
%

%
%

Next to in vivo experimentation, computational modeling is a valuable
approach to test interventions in neuronal circuits \textit{in silico} that avoids
the risks and costs associated with animal and human studies. Not only can it serve
as a translational tool to bring insights from neuroscience to clinical
applications, it can also serve as a safe environment for the design and
testing of DBS control algorithms \cite{modolo_model-driven_2011,beuter_closed-loop_2014,huys_computational_2016}.
Building on this approach, the primary aim of this thesis was to develop a detailed
computational model of the basal ganglia network that can be used to investigate the
pathophysiology of Parkinson’s disease and therapeutic interventions that aim to correct
it. To achieve these aims, a key requirement of the computational model was that it
captures the essential biophysical properties that shape pathological neuronal unit
and network activity, and the effects exerted by DBS upon it.
Moreover, the model should be implemented
in a computationally efficient manner that allows time-efficient simulation of the
network behavior under different parameter settings. The resulting model
can then be used to elucidate mechanisms of pathophysiology and the mechanism
of action of DBS, jointly at the cellular and network level.
The insights gained using the model can be used in a translational fashion, to improve
the efficacy of established DBS protocols and to guide the design of
novel stimulation protocols. %

%
A novel biophysically detailed model of the parkinsonian STN-GPe network was developed
that meets the requirements set forth in the aims. The model was implemented for
parallel execution across multiple processors. This enabled network simulations
on time scales of several hours, and demonstrates the feasibility of biophysically
detailed network modeling by leveraging distributed computer architectures.
%
The model was used to investigate the mechanisms underlying the emergence of
pathological beta-band oscillatory activity in the STN-GPe network,
which is correlated with akinetic-bradykinetic motor symptoms of Parkinson's disease.
%
It was shown that the STN-GPe feedback loop has an intrinsic susceptibility
to beta-band oscillations that is manifest in weak autonomously generated oscillations
that are strengthened by resonant interactions with cortical beta-band oscillatory
inputs. Resonance is established by phase-locking to cortical oscillatory
inputs in the dopamine depleted condition and is further amplified by striatal beta
inputs that promote anti-phase firing of the cortex and GPe.
The deleterious phase relationships between population spiking activity
in BG nuclei that enable strong phase locking are mediated by abnormal
synaptic transmission and modulation of ion channel conductances by dopamine depletion.
These changes affect the balance of excitation and inhibition in neurons,
their excitability, and their effective time constants of synaptic integration.
%
%
%
The detailed network model of the STN-GPe feedback loop was extended
using 3D neuron morphologies positioned inside a 3D reconstruction of the rat brain
and axon cable models were included for each cell. The resulting anatomically detailed
network model was used to elucidate mechanism of local cellular activation
by DBS, and of orthodromic and antidromic action potential propagation
during ongoing parkinsonian network activity. The results revealed mixed
excitatory/inhibitory effects of HFS DBS on STN somata depending on the orientations
and positions of neural elements, and strong antidromic activation of GPe neurons.
Moreover it was shown how the decoupling between somatic and axonal spiking
in STN neurons was sensitive to differential short-term depression in the
synapses of stimulated afferents.
%
Furthermore, based on the insights about deleterious phase relationships in PD,
a phase-locked DBS protocol was tested in the network
that disrupts phase locking in STN neurons by changing the relative phases of
afferent spiking activity. Phase-locked DBS consisted of low-frequency (20 Hz)
pulses delivered in anti-phase relative to 20 Hz cortical oscillatory inputs.
The suppression of beta-band oscillations in the LFP using this low-frequency
protocol suggests that phase-locked DBS could be an alternative to the
established HFS DBS that is less disruptive to ongoing network activity.
%
%
%

%
%
%

%
%
%
%
%
%
%
%
%
%

\section{Summary of contributions}
%

\subsection{Detailed STN-GPe network model}

The network model developed for this thesis was the first model of the STN-GPe
network in the basal ganglia that consists of biophysically detailed
multi-compartment neuron models. Despite the level of detail included
in the model, because of its parallel implementation it does not come with
a prohibitive cost in terms of execution time compared to networks consisting of
single-compartment neuron models.
%
%
%
%
%
%
%
%
The use of detailed multi-compartment neuron models allowed us to take into
consideration for the first time the interplay of active dendritic channel currents
and synaptic currents in shaping pathological unit and network activity.
These interactions were captured by delivering synaptic inputs to precise locations
in the dendritic trees where they activate locally expressed ion channels.
This revealed how strong burst firing modes in the STN were replaced by sparser
spiking and higher frequency beta-band activity depending on the balance of dendritic
excitation and inhibition, which regulated the availability of burst-generating
$Ca^{2+}$ channels.
%

%
%
%
%
%
The model showed an intrinsic susceptibility to beta-band oscillations
that was manifest in weak autonomously generated oscillations within the reciprocally
connected STN-GPe loop and in selective amplification of exogenous beta-band spiking
inputs near the network's endogenous oscillation frequency. Specifically, the frequency
response of the STN-GPe network showed a resonance peak in the beta-band range and the
frequency at which the peak occurred was determined by the ratio of excitation to
inhibition in the STN. As this ratio decreased, which is
supported experimentally by cortical de-afferentiation and strengthening of GPe synapses
after dopamine depletion, the resonance peak increased in magnitude and shifted toward the
lower beta-band range. Hence, the model hypothesizes a new mechanism by which
beta-band resonance arises in the STN-GPe loop.

The model also provided new insights into the role of phase relationships and
synaptic receptors in generating pathological activity. Autonomously generated oscillations
were strong only when GPe-STN synaptic inputs were dominated by
short time constant GABA\textsubscript{A}-receptor mediated currents while
GABA\textsubscript{B}-mediated currents were weak. However, strong bursting in STN
neurons required higher inhibition of the STN and the presence of slower
GABA\textsubscript{B} mediated currents. This indicates an important role for
GABA\textsubscript{B} receptors in the pathophysiology of Parkinson's Disease
in STN neurons. The weak autonomous oscillations in the presence of strong
GABA\textsubscript{B} currents and bursting did not support
a key pacemaker role for the STN-GPe network in the generation of beta-band
oscillations. Moreover, beta-band resonance in the STN-GPe network could be either suppressed
or amplified by beta-band striatal inputs to the GPe, depending on the
phase relationships in the network. Resonance was amplified by striatal beta
inputs that promoted anti-phase firing of the cortex and GPe, resulting in
maximum transient inhibition of STN neurons. This highlights the role of
phase relationships between populating spiking activity in shaping
the interaction of beta-band oscillations converging onto the STN-GPe loop
through different pathways.

%
The mechanism of oscillation in the model presented is consistent with that
of previous models where alternating phases of excitation and inhibition
in the delayed negative feedback loop consisting of STN and GPe give rise to
a sustained oscillation pattern (see Section~\ref{sec:ch3-disc:osc-mech-others}).
However, the detailed model further illustrates the role of precisely timed
excitation and inhibition in STN, orchestrated by the direct and indirect pathways,
in transiently de-inactivating somatic and dendritic ion channels to establish
strong phase-locking to its excitatory inputs.
Moreover, it highlights the sensitivity of the oscillation and bursting patterns
generated by the network to the excitation-inhibition balance in each population and synaptic current properties.
%

The model gives a mechanistic account of the generation of exaggerated
oscillatory activity in the STN-GPe network that is poorly understood
from experimental data alone, where the observable variables reflect
the state of the network more indirectly. %
Moreover, the computational model allows precise control over variables
that are hard to control in vivo or in vitro, such as the excitation-inhibition
balance in cells and the phase relationships between spiking inputs to the network.
This enables the investigation of the individual contributions of these
variables to network pathophysiology, which is difficult to do in vitro or in vivo.

\subsection{DBS model}
%
%
%
%
%
To leverage the biophysically detailed network model for the understanding
and design of DBS control algorithms, it was extended into a 3D anatomical
model. The 3D model was constructed by placing cell morphologies within
BG nuclei in an atlas-based reconstruction of the rat brain. Moreover,
it included multi-compartment models of the axonal projections between
the STN and GPe, and of the hyperdirect pathway originating in the cortex.
This was the first study where the distributed effects of DBS on
detailed neuronal morphologies were investigated during ongoing network
activity. Importantly, this allowed us to characterize for the first time the
effect of stimulus-locked action potentials in multiple fiber bundles,
propagating orthodromically and antidromically along axons in the network
and disrupting intrinsic population spiking both downstream and upstream
of the stimulation site.
%
%
%
%

%
%
%
%
%
Simulation results indicated that antidromic recruitment of GPe neurons
via the GPe-STN projection could play an important role in the network
effects of DBS. Furthermore, the results further clarified seemingly contradictory
data in the literature reporting either excitatory or inhibitory effects
of DBS on somatic spiking activity. Entrainment of STN somata to HFS DBS
was variable, and cells were either excited or inhibited, depending on
local neurite orientations and location with respect to the electrode. Moreover, there
were different degrees of dissociation between somatic and axonal spiking
between neurons. The degree of dissociation was shown to
depend on differential short-term depression in excitatory and inhibitory
afferents to the STN, confirming existing hypotheses based on experimental
recordings. Finally, the model gave new insights into the mechanism
of closed-loop, phase-locked stimulation protocols that are
emerging as candidates to replace traditional high-frequency DBS.
Application of a low-frequency stimulation protocol that was phase-locked to the
cortical beta-band oscillations entering the network revealed an optimal
phase of stimulation for the suppression of beta-band oscillations in the LFP.
Specifically, stimulating 180 degrees out-of-phase with respect to the
onset of cortical bursts dispersed STN spiking throughout the oscillation
period resulting in the suppression of phase-locking.

Hence, the model yielded new insights into the possible mechanisms
of phase-locked stimulation, which is a promising candidate therapy for
Parkinson's disease that is actively being investigated \cite{holt_phase-dependent_2019}.
Moreover, the model and the resulting insights could be used in future
studies to optimize the control parameters and electrode placements
for phase-locked stimulation as well as other closed-loop DBS control algorithms

\subsection{Reduced STN-GPe network model}
%
%
%
%
%
To capture the biophysical properties that determine pathological neuronal activity
in the STN-GPe network and its interaction with DBS, morphological neuron models
were assumed to provide a sufficient level of detail. However, such models
consist of a large number of state variables, which burden them with high computational
complexity and potentially a superfluous level of detail.
Because the aim was to create a model that provides the \textit{essential}
level of detail to capture these properties, a reduced network model was
developed based on the original detailed model. The reduced model
had a lower number of state variables, determined by the number of neuronal
compartments, and is more suitable for the exploration of parameter
spaces in the network description and DBS protocols.
%
Reduction of the network model followed a bottom-up approach, in contrast
to established methods for reducing the number of state variables to represent
neuronal network activity which are top-down, e.g. mean field models.
The approach followed was to substitute reduced morphology models for the
original cell models in the network. While the effect of morphology reduction
on single-cell responses is studied in various cell types, this was the first
study were its effect on network-level activity is investigated. Results showed that,
while individual cell responses and synchronization properties could be preserved, the
network-level activity and its response to synchronous inputs were subtly
altered. Autonomous oscillations in the network, and resonance with oscillatory
inputs were preserved. However, bursting was less pronounced, the strength
of oscillations was increased, and there was a frequency shift in the resonance
peak exhibited by the network. This was related to differences in the processing
of synaptic inputs in the reduced neuron models as a results of changes
in the electrical structures of the cells. The reduced network model can
be extended in the future to incorporate the effects of DBS based on the
findings in Chapter~\ref{ch4:dbs-model}.

%
%
%

\section{Modeling approach}
\label{sec:ch6-discussion-modeling-approach}
%
The modeling approach followed in this thesis was that of biophysically
detailed, multi-compartment cable neuron models. The rationale for choosing
this methodology was twofold. First, it allowed to capture the spatially distributed
effect of the electric field on neuron morphologies, which depends strongly
on second-order spatial gradients of the field along neurite elements \cite{rattay_basic_1999}
and possible interactions with dendritic voltage-gated ion channels \cite{rattay_which_2010}.
Second, modeling dendrites allows to take into account somato-dendritic interactions
occurring in neurons as a result of the interplay between intrinsic and
synaptic currents. These effects are known to underlie pathological bursting
in STN neurons \cite{gillies_membrane_2005,beurrier_subthalamic_1999,otsuka_excitatory_2001,song_characterization_2000},
and influence the processing of synaptic inputs \cite{hanson_sodium_2004,chan_hcn2_2004},
as well as the synchronization properties of STN and GPe neurons \cite{schultheiss_phase_2010,farries_phase_2012}.

%
Because of the biophysically detailed modeling approach, and the fact that
most available experimental data about basal ganglia pathophysiology
comes from rodent experiments, models were developed to represent the rat brain.
This allowed the use of reconstructed neuron morphologies, detailed anatomical brain atlases,
and experimentally validated biophysical parameters for the neuron and electrical models.
The use of animal models for the investigation of PD pathophysiology
and therapies is standard practice.
Because of the evolutionary conservation of basal ganglia nuclei in vertebrates
\cite{grillner_basal_2016}, the expression of Parkinsonian pathophysiology
after dopaminergic lesioning, and many motor and non-motor symptoms are
replicated between species \cite{blesa_classic_2012}.
Moreover, evidence suggests that the mechanism for generation of pathological
oscillatory activity is conserved between rodents and primates \cite{sherman_neural_2016}.
If this is indeed the case, then insights into the mechanisms of pathological
synchronization, as presented in this thesis, should remain valid.
Although differences in brain anatomy between species require different
electrode configuration and placement, stimulation protocols designed to
desynchronize neural populations based on these insights should be
expected to be transferable.
%
%


%
%
%
%
%
%
%
The approach to model fitting was to start from available cell models that
were already fitted to electrophysiology data, and subsequently constrain the resulting
network model using available data about synaptic physiology and connection patterns
reported in the neurophysiology literature. This was followed by hand-tuning
of synaptic weights to obtain mean population firing rates in agreement with experimentally
reported data in rats.
This hand-tuning procedure is \textit{ad hoc}, in contrast to established numerical
optimization routines, and is used widely in the field of computational neuroscience.
The main obstacles to numerical optimization are a lack of high-quality datasets
available for model fitting with standardized descriptions of experimental conditions
and the computational resources required to evaluate large network models in each
step of an optimization procedure. In addition, little is known about the convergence
properties of optimization procuedures for network models with high-dimensional
parameter landscapes consisting of parameters that have heterogeneous non-linear
effects on network dynamics and cost functions that balance many, possibly competing objectives.
%
%
The lack of such high-quality datasets in combination with the lack of suitable
optimization routines with guaranteed convergence properties, and of established
regularization techniques leads to a high risk of overfitting. This problem is
compounded in detailed models by the higher number of free parameters, particularly
those that cannot be constrained due to a lack of experimental data, leading
to higher model variance.
%
%
%
On the other hand, biophysically detailed models have the
advantage that their variables have a direct relationship to biophysical
quantities and can therefore more easily be compared to experimental data.
Because of the explicit modeling of biophysical processes, unrealistic parameter
values may result in non-biological behaviors occurring only in vitro, such as depolarization
block occurring when excitatory inputs are too strong. Hence, unlike in abstract
neuron models, unrealistic parameter values can be identified and rejected more
easily.
%
This property reduces the likelihood that the models presented here
are severely overfit with respect to synaptic weights, the primary free parameters.
Moreover, parameter sweeps were performed where possible to inspect the network
behaviour in the local region of the parameter landscape, along axes corresponding
to synaptic weight parameters belonging to one projection type. At least
along these axes, the networks showed gradual transitions in their behavior
indicating that hand-tuned parameters did not correspond to unstable local minima.
Similar controls were carried out by altering the randomly-generated connection matrices.
%
%
%
%
%
However, network activity patterns as shown in this thesis could be highly sensitive to other parameters.
The introduction of structured connectivity patterns for example can strongly influence
oscillation patterns, as shown in an early model of the STN-GPe network~\cite{terman_activity_2002}.
A second important source of variability in network behaviour is the neuron models used,
and how well their biophysical properties capture the in vivo distribution of neuron responses
within a population.
This is an active area of research, and it is well known that multi-objective optimization methods
typically used for neuron model fitting can generate many candidate parameter sets that score equally well
on the objective function \cite{van_geit_automated_2008}, but show difference responses
to novel stimulation patterns. Similarly, these equally performing model candidates may also show
different synchronization properties and lead to different network activity patterns.
%
%
%
%

%
%
%
%
%

%
%
%
%
%
%
%
%
%
%

\section{Future research}
%
%


The network model used as the basis of the three modeling studies in this thesis
consists of the two interconnected nuclei where parkinsonian pathophysiology is
most prominent and where DBS has proven to be the most effective. Although
the network model captured a high degree of biophysical detail, many directions
can be explored to extend the model for further optimization and design of
DBS protocols. %

\subsection{Extending the STN-GPe network model}
%
%
%
%
%
%
%
%
%
In order to understand the relation between neural activity patterns and parkinsonian
motor symptoms, the output stages of the basal ganglia projecting to motor command
structures in the brain stem and cortex should be modeled. Only when the link
between basal ganglia activity and motor processing can be made, can stimulation
protocols truly be optimized in terms of their ability to restore normal motor-related activity.
To achieve this, a functional dimension should be added to network models
by imposing relationships between the inputs and outputs
of the network that represent motor-related activity and processing.
This could be achieved based on neural recordings at the input
and output stages, and using learning rules for spiking neural networks
that modify synaptic weights \cite{abbott_building_2016} to learn
input-output relationship that satisfy particular motor requirements.
Such a functional understanding would enable
better strategies for neuromodulation, incorporating specific cost functions
in terms of the activity patterns that should be restored in the network.
One major challenge is that learning rules for biological spiking neural networks
are not well understood, and that they would have to optimize multiple
cost functions. Besides input-output relationships, biological
constraints on activity patterns exhibited by the network should be satisfied.
To address this challenge, learning methods for complex cost functions in structured
architectures developed for non-spiking neural networks are being reconciled
with those for spiking networks \cite{marblestone_toward_2016,abbott_building_2016,depasquale_using_2016}.
This should enhance the understanding of the relationship between
parkinsonian pathophysiology and the disruption of motor processing,
and should lead to the development of more effective stimulation protocols. %

%
A second way to extend the presented network models, is to integrate
the multiple feedback loops in the basal ganglia-thalamocortical network
that are thought to play a role in the generation and propagation of pathological
oscillations. In particular the thalamo-cortical loop is considered to be a key
feedback loop \cite{pavlides_computational_2015,reis_thalamocortical_2019} and
links the core basal ganglia nuclei of the GP and STN to the cortex,
which is believed to drive the generation of pathological beta-band oscillations
\cite{sherman_neural_2016} and is key to its suppression \cite{li_resonant_2007,li_therapeutic_2012}.
Modeling the feedback loops rather than including cortical beta-band activity
as an external input will allow investigation of the effect of multiple
circuit resonances in the basal ganglia and how these are affected by DBS.



\subsection{DBS model and protocols}
%

%
%
%
%
%
%
%
%

%
%
%
%
%
%
To improve efficacy of existing high-frequency DBS protocols, the model presented
in Chapter~\eqref{ch4:dbs-model} can be used to investigate optimal placement of
electrodes and contact activation,
in combination with a more detailed volume conductor model taking into account
different tissue properties. This would allow more precise targeting of neuronal
elements and investigation of their differential contributions to the modulation %
of network activity.

Furthermore, the analysis of STN-GPe network activity in Chapter~\ref{ch3:detailed-model}
and the application of phase-locked stimulation in Chapter~\ref{ch4:dbs-model} suggested
that altering phase-relationships between basal ganglia nuclei is a promising
method to attenuate pathological oscillations. Indeed, preliminary data in rats
shows that phase-locked stimulation in the STN can desynchronize local neuronal
activity within several stimulation cycles \cite{holt_phase-dependent_2019}.
Moreover, multi-site HFS designed to disrupt phase relationships within populations
is more effective that continuous high-frequency DBS in parkinsonian monkeys \cite{tass_coordinated_2012}.
The current model could be used to investigate multi-site phase-locked
stimulation protocols that alter phase-relationships by activating different
neuronal elements, with mixed orthodromic and antidromic propagation effects,
in a way which would not be possible with simpler single cell models.


\subsection{Data-driven biological modeling}
%
%
%
%
%
%
%
%
%
%
%
%
%
%
%
%
%
%
%
A key issue that needs to be addressed in future models is that of biological variability
in cellular properties and, in particular, how it is affected by dopamine depletion
in Parkinson's disease. First, there is considerable heterogeneity in
morpho-electric cell types in the brain \cite{ramaswamy_neocortical_2015,zeng_neuronal_2017},
including within basal ganglia nuclei \cite{mallet_dichotomous_2012,sharott_different_2009,sharott_population_2017}.
Subtypes of cells can have different morphological properties,
ion channel expression, intracellular signaling pathways, and
take part in fine-grained connection motifs within brain nuclei.
All these properties contribute to variability in responses and firing
patterns in these neurons, which reflect their different functional
roles in the network. However, network models more often than not consist of
identical copies of the same cell model within nuclei with uniform connection motifs.
If this diversity in cellular properties is not captured, the emergent
activity patterns and the network's responses to interventions may not
reflect critical features of the biological network in vivo.
%
The problem of variability can be addressed using high-throughput
electrophysiology methods such as automated patch-clamp, and optimization
procedures for the automatic construction of neuron models based on large datasets \cite{gouwens_systematic_2018,pozzorini_automated_2015}.
%
%
Large-scale simultaneous extracellular recordings could also be used to validate
the network model behavior. In this thesis, spiking patterns in the network
were validated based on reported population firing rates and manual tuning
of synaptic connection strengths. However, using the aforementioned learning
rules that adapt synaptic weights based on complex %
cost functions, this process could be automated and the fit could be improved
by including complex measures based on signal and spike train analysis.

A second area where data-driven modeling will be key to the success of
network models in designing anti-parkinsonian interventions is the mapping of
physiological changes in the basal ganglia following dopamine depletion,
and capturing them in models. Dopamine depletion affects a variety of
signaling cascades involved in regulating ion channel function and the expression
of membrane-bound proteins, synaptic remodeling, and dendritic branching.
However, these changes are often not characterized in a robust quantitative manner
but instead known through sparse observations in unclear experimental conditions.
Larger datasets and the development of standards for experimental data collection
and reporting can help to quantify the effects of these changes on cellular dynamics.
Network models can then help to relate this to large-scale network dynamics and motor processing.
%

In summary, data-driven modeling is essential to accelerate the iterative cycle of
model-based hypothesis generation, hypothesis testing, and model refinement.
The speed up of this process will be facilitated by the development and adoption
of new technologies for high-throughput data collection and processing in quantitative biology.

\section{Closing remarks}
%
%
%
%
%
%

The work presented in this thesis demonstrates that valuable insights about neuronal
network dynamics and network mechanisms of neurostimulation can be obtained
using biophysically detailed models. Despite the effort to capture the wealth of
biophysical data available about BG physiology and neurobiology, the models
can be extended to include additional data and biological processes that could
reveal new dynamics arising from the interplay of biophysical processes at different scales.
In particular, the role of biochemical signaling cascades and the dynamics of
neurotransmitter release, receptor binding, and synaptic integration in different
classes of synapses remain under-explored.
%
Given the  advent of big data platforms as pioneered by the
\href{https://www.humanbrainproject.eu}{Human Brain Project} and
\href{https://portal.brain-map.org/}{Allen Brain Atlas}, incorporation of this
level of detail is possible on a scale that has not been feasible before now.
Moreover, biophysically detailed modeling will undoubtedly become more appealing
in the future, with the availability of next-generation simulators
that exploit parallel computing architectures \cite{akar_arbor_2019,kumbhar_coreneuron_2019}.
The work presented in this thesis can be a stepping stone for researchers wishing to
go down this path to develop future therapies for managing the symptoms of neurological disorders.

%
%
%
%
%
%
%
