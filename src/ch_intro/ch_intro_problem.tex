%
%
%
%
%

%
%

%
%
%
%
%
%
%
%
%
%
%

%
%
%
%
%
%
%
%
%
%
%
%
%
%

%
%
%
%
%
%


%
%

%
%
Parkinson's disease (PD) is the second most prevalent neurodegenerative disease,
affecting approximately 1\% of the global population over 65 years of age and 5\%
of the population over the age of 85 \cite{deRijk1997prevalence}.
It is associated with several motor and neuropsychiatric symptoms \cite{jankovic_parkinsons_2008},
with the primary motor symptoms characterized by muscular rigidity leading to
bradykinesia and akinesia, as well as tremor and postural instability.
%
Deep brain stimulation (DBS) is a clinically effective and reversible therapy
to manage the motor symptoms of PD that is most commonly prescribed in late-stage PD
after significant resistance to pharmacological treatment has developed
%
\cite{pizzolato_deep_2012,bronstein_deep_2011,weaver_bilateral_2009,benabid_deep_2009}.
%
%
%
DBS is also being considered as a viable treatment option for early-stage PD
motor symptoms \cite{schuepbach_neurostimulation_2013,hacker_deep_2015} as well as
other neurological and psychiatric disorders~\cite{lyons_deep_2011,krack_deep_2010}.
%
%

%

%
%
%
Despite its clinical success and widespread adoption as a standard of care,
DBS therapy still has several important shortcomings. On average, it offers
only 40-50\% improvement in motor symptoms as assessed by the unified
Parkinson's disease rating scale (UPDRS)
\cite{chen2006deep,deuschl_randomized_2006,schuepbach_neurostimulation_2013,rabie_improvement_2016}.
Moreover, the indications for DBS in PD are primarily
akinetic-bradykinetic symptoms and tremor whereas the effectiveness is
not clinically established for the treatment of axial symptoms such as freezing of gait,
and for mood disturbances and cognitive symptoms \cite{eisenstein_acute_2014,merola_impulse_2017,huang_deep_2018}.
Furthermore, tolerance to DBS therapy is developed in a subset of patients
~\cite{kumar_long-term_2003,houeto_failure_2000}, necessitating frequent
reprogramming of the stimulation parameters and different stimulation-induced
side effects occur with varying rates of incidence~\cite{guehl_statistical_2007,benabid_deep_2009}.
%

\section{Thesis aims}

%
%
Overall, the shortcomings of DBS are related to a lack of understanding of its underlying
mechanisms. Although the direct cellular effects of DBS are better understood \cite{chiken_mechanism_2016,mcintyre_deep_2016,jakobs_cellular_2019}, the picture
changes when neurons are embedded in a network and it is still unclear how DBS
impacts the basal ganglia in terms of its network-level activity.
This is of particular interest given that DBS is effectively a non-local intervention that
affects activity patterns throughout the basal ganglia, both downstream and upstream of the stimulation site
\cite{dorval_deep_2008,li_resonant_2007}. Hence a better understanding of how DBS
interacts with BG network activity is needed in order to overcome its limitations
and improve control over Parkinsonian symptoms and stimulation side effects.
%
%

%
%
%
Computational models provide a valuable tool to investigate the mechanism of action
of DBS at both the cellular and network level. Mean-field models of neuronal
populations and single-compartment neuron models are commonly used to answer
questions about network synchronization, whereas multi-compartment cable models
are used to study intracellular biophysical properties of individual neurons
and the interaction with extracellular fields.
However, it is becoming clear that macroscopic synchronization
properties and features of neuronal spiking activity relevant to PD pathophysiology,
such as exaggerated bursting, can depend on neuron's subcellular properties
\cite{chan_hcn2_2004,gillies_membrane_2005,schultheiss_phase_2010,farries_biophysical_2012,crook_dendritic_1998,goldberg_response_2007}.
%
The effect of electrical stimulation on a neuron's output activity can vary depending
on these same subcellular properties, which include non-uniform ion channel distributions,
dendritic branching patterns, and the distribution of synapses over the cell
\cite{rattay_which_2010,yousif_spatiotemporal_2012,steiner_connectivity_2019}.
%
Despite their importance, computational models of the basal ganglia
focusing on pathophysiology and network interactions with DBS have so far not
included these properties.
%
Moreover, there is now clear evidence that DBS is a network phenomenon \cite{mcintyre_network_2010}
therefore network interactions must also be considered.

%
Therefore, the first aim of this thesis was to formulate a model of the basal ganglia
network that captures both the essential biophysical properties that shape
pathological neuronal unit and network activity, and the effects exerted by DBS
upon the network. The second aim  was to investigate the
local cellular and wider network-level effects of DBS upon the basal ganglia
with the aim of better understanding its mechanism of action so that future
therapies may be improved. Finally, the third aim of the thesis was to
investigate what elements of the detailed biophysical model
could be dispensed with to arrive at a compact model that retains its essential
biophysical properties but is more computationally efficient and therefore more
suitable for the exploration of parameter variations in the model.
A reduced model enables the development of new closed-loop stimulation
algorithms for DBS by exploring the parameter space of the network,
electrode configuration, and stimulation parameters in a computationally efficient
manner.
%

%
Because experimental data about basal ganglia pathophysiology and neurobiology
in the literature and available through public databases derives largely
from rodent models of PD, the computational models developed for this thesis
represent the rat brain. This allowed the use of experimentally validated
biophysical parameters and existing neuron models, increasing biological realism
of the models. Animal models are widely used to investigate PD pathophysiology
and the effects of therapeutic interventions %
because basal ganglia organization and physiology, and the way PD motor symptoms are
manifested are very similar between species \cite{grillner_basal_2016,blesa_classic_2012}.
Hence the insights into mechanisms of pathophysiology and mechanism of action of DBS
obtained using the developed models are expected to be transferable to PD patients
(see Section~\ref{sec:ch6-discussion-modeling-approach}).

\section{Thesis overview}
%
The scientific literature relevant to addressing these problems is reviewed
in Chapter~\ref{ch2:lit-review}. First Parkinson's disease is introduced and
subsequently its pathophysiological features are discussed from the
perspective of the cellular and subcellular level. Then, pathophysiology
at the network-level is discussed with a focus on pathological synchronization
of neuronal activity and the relation to Parkinsonian motor symptoms.
Experimental and modeling studies focusing on the mechanisms of DBS are reviewed,
and a final section reviews different approaches to computational modeling.

%
A new model of the subthalamo-pallidal network consisting of biophysically
detailed cell models is presented in Chapter~\ref{ch3:detailed-model}.
This network consists of a reciprocal loop between two nuclei of the
basal ganglia that is considered key for the generation of pathological
activity patterns and their disruption by DBS. The model is used to
investigate how pathological oscillations and bursting can arise in this
network through network interactions between its populations, and through
the interaction between synaptic and intrinsic currents in dendritic structures.
The model shows how phase locking of subthalamic and pallidal neurons and
exaggerated bursting in subthalamic neurons can arise from the interaction
of these currents when the balance of excitation and inhibition is changed
and how phase locking is amplified under specific phase relationships between
cortical and striatal beta-band inputs.
%
%

%
%
%
%
%
%
%
%
%
In Chapter~\ref{ch4:dbs-model}, the cellular and network effects of DBS
are investigated using the biophysically detailed model presented in the
previous Chapter~\ref{ch3:detailed-model}. This is the first model where
neurons are simultaneously
affected by the applied electric field along their entire morphologies
and embedded in a network model, so that both efferent and afferent axons
to each neuron are susceptible to stimulation. This enables the characterization
of cellular and network effects of DBS while stimulus-locked action potentials can
propagate both orthodromically and antidromically along each axon, with synapses
at the terminals exhibiting short-term plasticity affected by this stimulation.
%
Using the model, it is demonstrated how the dissociation between
somatic and axonal stimulus-locked spiking in neurons is determined
by the degree of short-term depression in excitatory and inhibitory synapses.
%
Furthermore, the model is used to investigate the effects of low-frequency
stimulation phase-locked to the frequency of beta-band oscillations in
cortical inputs to the network. By applying phase-locked DBS, it is shown
how the suppression of oscillations in the network depends on the phase
of stimulation by altering the relative phases of population spiking activity
in the network. Simulation results reveal an optimal phase of stimulation
for the suppression of beta-band oscillations which occurs when pulses
are applied 180 degrees out-of-phase with respect to the onset of cortical bursts.

%
The biophysically detailed neuron models used in the network models presented in
Chapters~\ref{ch3:detailed-model} and \ref{ch4:dbs-model} capture a high level of
physiological detail and can simulate characteristic behaviours of subthalamic nucleus
(STN) and external globus pallidus (GPe) neurons.
However, the large number of neuronal compartments within the individual neuron
models results in a high computational complexity which make the network model
computationally expensive to simulate and unwieldy for the exploration of parameter
variations in the network or stimulation protocol.
%
To address this problem, in Chapter~\ref{ch5:reduced-model} a derived network model
is presented where reduced morphology neuron models are substituted for the original,
detailed neuron models. First, a model reduction procedure is presented based on
the collapsing of dendritic branches into equivalent cable structures. The reduction
procedure is shown to conserve somatic responses and synchronization properties
of detailed single neuron models.
Then the resulting network model is used to show how key synchronization properties
and neuronal firing patterns of the detailed network model are conserved, while differences
are related to changes in the electrical properties of neurons introduced by the
reduction procedure.

%
Finally, in Chapter~\ref{ch6:conclusions} the results presented in this thesis,
its conclusions and contributions toward the understanding of DBS for the
treatment of Parkinson's Disease are summarized. Future directions based on
this research are also discussed.